\documentclass{article}
\usepackage[utf8]{inputenc}
\usepackage[turkish]{babel}

\title{Oda Kuralları}
\author{ODTÜ Bilgisayar Topluluğu}

\begin{document}
\maketitle

\section{Yemek ve çay-kahve hakkında}

Yiyecek ve içecek tüketimi odada en çok problem yaratan konulardan birisidir. Bunun iki sebebi var: birincisi, özellikle odanın kalabalık olduğu zamanlarda yemek kokuları odadaki diğer üyeleri rahatsız etmektedir. İkincisi, odada bir şeyler içmek her yerinde elektronik alet bulunan odamızda büyük bir risk faktörü haline geliyor. Bu yüzden üyelerimizin odada özellikle odanın kalabalık olduğu zamanlarda yiyecek içecek tüketmekten olabildiğince kaçınmasını ve çok gerekli durumlarda diğer üyelerin rızasını aldıktan sonra odada bunları tüketmesini öneriyoruz.

\begin{enumerate}
	\item Su ısıtıcısının fişini kullandıktan sonra çekin.
	\item Islak mendil, plastik kaşık ve çatal, çeşitli soslar ve tuzları masada bırakmayın. Hepsinin dolapta yeri var. 
	\item Odaya kendi kupanızı getirebilirsiniz. Plastik şişe ve bardak yerine kendi kupanızı kullanmanızı tavsiye ediyoruz. Kullandıktan sonra kupanızı mutlaka yıkayıp kuruması için uygun yere bırakın.
\end{enumerate}

\section{Odanın temizliği ve düzeni}

Bütün üyelerimizin topluluk odasından yararlanma hakkı olduğu gibi bütün üyelerimizin odanın temizliği ve düzeni konusunda birtakım sorumlulukları vardır. Odanın temizliğini ve düzenini sağlamak herhangi tek bir üyenin, sadece yönetim kurulunun veya denetleme kurulunun görevi değildir. Bütün üyelerin odanın temizliğinde gerekli hassasiyeti göstermesini ayrıca gerekli gördüğünde kendilerinin de odanın temizliği için toz almak, eşyaları toplamak gibi küçük işleri yapmalarını bekliyoruz.

\begin{enumerate}
	\item Özellikle odadaki kişi sayısı arttıkça periyodik olarak pencereyi açıp odayı havalandırmayı unutmayın. Dışarıdan gelen insanlar için odanın havasız olması çok rahatsız edici oluyor. Odadan çıkan son kişi pencereyi mutlaka kapatmalıdır.
	\item Odaya çöp bırakmayın. Burada en çok dikkat edilmesi gereken şeyin masanın üzerine bırakılan su şişeleri olduğunu belirtmemizde fayda var. \textbf{Lütfen su şişelerinizi masada bırakıp gitmeyin.} Ayrıca gözünüze erişen çöpleri size ait olmasa bile kapının önündeki çöp kutusuna atabilirsiniz. Ayrıca koltuk ile kapı arasında paspas ve süpürge ile faraş var. Gerektiğinde kullanmaktan korkmayın.
	\item Çantalarınızı koltuk ve sandalyelere bırakmayın. Masa altına duvar kenarındaki yerlere veya dolabın yanına bırakabilirsiniz. Eğer yere koymaktan rahatsız oluyorsanız dolabın yanındaki sehpaya koyabilirsiniz.
	\item Montlarınızı kapının arkasına asın.
	\item Eşyalarınız masada diğer insanların alanını kısıtlamayacak şekilde konumlandırın.
	\item Eğer odadan uzun bir süreliğine ayrılacaksanız eşyalarınızı toplayın. Masayı ve koltuğu gereksiz yere meşgul etmeyin.
\end{enumerate}

\section{Birlikte Yaşam}
Her üyenin odadan rahatça yararlanabilmesi için son olarak bahsetmek istediğimiz birkaç kural daha var.
\begin{enumerate}
	\item Önceki iki bölümde bahsedilen kurallara ve Code of Conduct'a uyun.
	\item Odada ders çalışan insanlara saygı duyun. Odanın o anki durumunu sezebilin.
	\item Diğer üyelerden izin almadan odada sesli yayın yapmayın. Ses seviyesi önemli değil.
\end{enumerate}
	
\end{document}
