\documentclass{article}

\usepackage[turkish]{babel}


\title{Oda Kuralları}
\author{ODTÜ Bilgisayar Topluluğu}

\begin{document}
\maketitle



\section{Yemek ve Çay-Kahve Hakkında}

Yiyecek ve içecek tüketimi odada en çok problem yaratan konulardan birisidir.
Bunun iki sebebi var: Birincisi, özellikle odanın kalabalık olduğu zamanlarda
yemek kokuları odadaki diğer üyeleri rahatsız etmektedir. İkincisi, odada içecek
tüketmek çoğunlukla elektronik alet bulunan odamızda büyük bir risk faktörü haline geliyor. Bu yüzden üyelerimizin odada özellikle odanın kalabalık olduğu
zamanlarda yiyecek içecek tüketmekten olabildiğince kaçınmasını ve çok gerekli
durumlarda ancak diğer üyelerin rızasını aldıktan sonra odada bunları
tüketmesini öneriyoruz.
\begin{enumerate}
    \item Islak mendil, plastik kaşık ve çatal, çeşitli soslar ve tuzları masada bırakmayın.
    \item \textbf{Odaya kendi kupanızı getirebilirsiniz.} Plastik şişe ve bardak yerine kendi kupanızı kullanmanızı tavsiye ediyoruz. Kullandıktan sonra kupanızı mutlaka yıkayıp kuruması için uygun yere bırakın.
\end{enumerate}


\section{Oda Temizliği ve Düzeni}
Bütün üyelerimizin topluluk odasından yararlanma hakkı olduğu gibi bütün üyelerimizin odanın temizliği ve düzeni konusunda sorumlulardır. Odanın temizliğini ve düzenini sağlamak sadece bir üyenin, yetkili bir kişinin, yönetim kurulunun veya denetleme kurulunun görevi değildir. Bütün üyelerin odanın temizliğinde gerekli hassasiyeti göstermesini ayrıca gerekli gördüğünde kendilerinin de odanın temizliği için toz almak, çöpleri atmak, eşyaları toplamak ve gerekli yerlerine koymak gibi işleri yapmalarını bekliyoruz.
\begin{enumerate}
    \item Herhangi bir hastalık durumunda, diğer üyelere bulaştırmamak adına, oda içerisinde maske takmaya özen gösterin.
    \item 	Eğer oda içerisinde oturacaksanız, odada büyük kalabalık oluşturmadan oturabilirsiniz. Kısa süreli işlerde \emph{-eşyamı alıp çıkacağım gibi-} bu sınır aşılabilir.
    \item 	Odada bulunduğunuz süre boyunca pencereyi açık tutmaya özen gösterin. Bazı durumlarda -soğuk hava, yağmur/kar yağışı gibi- pencereyi kapatmanız gerekebilir. Bu durumlarda tam ve buçuklu saatlerde (15.00, 15.30 gibi) \textbf{en az 5 dakika} odayı havalandırmaya özen gösterin. Odadan çıkan son kişi pencereyi mutlaka kapatmalıdır.
    \item 	Odaya çöp bırakmayın. Burada en çok dikkat edilmesi gereken şeyin masanın üzerine bırakılan su şişeleri olduğunu belirtmemizde fayda var. \textbf{Lütfen su şişelerinizi masada bırakıp gitmeyin.} Ayrıca gözünüze erişen çöpleri size ait olmasa bile oda içindeki ya da koridordaki çöp kutusuna atabilirsiniz. Son olarak odada süpürge ile faraş var. Gerektiğinde kullanmaktan çekinmeyin.
    \item \textbf{Çantalarınızı koltuk ve sandalyelere bırakmayın.} Masa altına duvar kenarındaki yerlere veya dolabın yanına bırakabilirsiniz.
    \item Montları kapının önündeki askılığa asın. 
    \item Eşylarınızı masada diğer insanların alanını işgal etmeyecek şekilde konumlandırın.
    \item Eğer odadan uzun bir süreliğine ayrılacaksanız eşylarınızı toplayın. Masayı ve koltuğu gereksiz yere meşgul etmeyin.
\end{enumerate}

\section{Birlikte Yaşam}
Her üyenin odadan rahatça yararlanabilmesi için son olarak bahsetmek istediğimiz birkaç kural daha var.
\begin{enumerate}
    \item 	Önceki iki bolümde bahsedilen kurallara ve \textbf{Code of Conduct}’a uyun.
    \item Odada ders çalışanlara saygı duyun ve rahatsız etmeyin. Odanın o anki durumunu sezebilin
    \item Ses seviyesinden bağımsız bir şekilde diğer üyelerden izin almadan odada sesli yayın yapmayın.
\end{enumerate}

\end{document}
