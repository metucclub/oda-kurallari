\documentclass{article}

\usepackage[turkish]{babel}


\title{Oda Kuralları}
\author{ODTÜ Bilgisayar Topluluğu}

\begin{document}
\maketitle



\section{Yiyecek ve İçecek Tüketimi Hakkında} 

\hspace{5mm} Odada her türlü yiyecek tüketimi \textbf{kesinlikle} yasaktır. \\

Alkollü içkilerin odada tüketilmesi \textbf{kesinlikle} yasaktır. Kokusuz ve basit içecekler içilebilir (çay, kahve, oralet vs). Sıcak içecekler için sıcak su sebilden temin edilmelidir. Kettle kullanımı aksi belirtilmediği sürece \textbf{yasaktır.} Kendi kupanızı odada barındırabilirsiniz, ancak herkes kendi kupasının güvenliğinden ve temizliğinden sorumludur. \\

Odadaki sarf malzemelerinden (ıslak mendil, kağıt havlu, karıştırıcı vs) herhangi birinin bittiğini fark ettiğinizde bir YK'ya haber vermelisiniz. YK gereken malzemeyi tedarik edecektir. \\

Odadan çıkılırken, sebilin fişi çekilmeli ve tüm fonksiyonları arkadaki düğmelerinden kapatılmalıdır. Sebilin sıcak su fonksiyonu sadece ihtiyaç duyulduğunda açılmalı ve iş bitince kapatıldığından emin olunmalıdır. \\


\section{Oda Temizliği ve Düzeni}
\hspace{5mm}Bütün üyelerimiz topluluk odasından yararlanma hakkına sahip olduğu gibi odanın temizliği ve düzeni konusunda da sorumludur. Odanın temizliğini ve düzenini sağlamak sadece bir üyenin, yetkili bir kişinin, yönetim kurulunun veya denetleme kurulunun görevi \textbf{değildir}. Odayı kullanan \textbf{bütün üyeler} odanın temizliğinde gerekli hassasiyeti göstermelidir ve ayrıca gerekli gördüğünde kendileri de odanın temizliği için toz almak, çöpleri atmak, eşyaları toplamak ve gerekli yerlerine koymak gibi işleri yapmalıdır. Odayı kullanan her \textbf{aktif} üye, YK tarafından belirlenen haftalık oda temizlik çizelgesinde en az bir işi yapmalıdır.
\begin{enumerate}
    \item Herhangi bir hastalık durumunda, diğer üyelere bulaştırmamak adına, oda içerisinde maske takmaya özen gösterin.
    \item 	Odada bulunduğunuz süre boyunca pencereyi açık tutmaya özen gösterin. Bazı durumlarda -soğuk hava, yağmur/kar yağışı gibi- pencereyi kapatmanız gerekebilir. Bu durumlarda tam ve buçuklu saatlerde (15.00, 15.30 gibi) \textbf{en az 5 dakika} odayı havalandırmaya özen gösterin. 
    \item Odadan çıkan son kişi pencereyi mutlaka kapatmalıdır.
    \item Masadaki işiniz bittikten sonra \textbf{masayı ıslak mendille, gerekirse kolonya da kullanarak, silin}; temiz ve kuru bırakın. Arkanızda silgi çöpü, kırıntı gibi kalıntılar bırakmayın.
    \item Odaya çöp bırakmayın. Burada en çok dikkat edilmesi gereken şeyin masanın üzerine bırakılan su şişeleri olduğunu belirtmemizde fayda var. \textbf{Lütfen su şişelerinizi masada bırakıp gitmeyin.} Ayrıca gözünüze erişen çöpleri size ait olmasa bile oda içindeki ya da koridordaki çöp kutusuna atabilirsiniz. Son olarak odada süpürge ile faraş var. Gerektiğinde kullanmaktan çekinmeyin.
    \item Odadaki çöp kutusunu dolu görmeniz halinde üzerine daha fazla çöp dizmek yerine kattaki çöpe atın ve yeni bir çöp poşeti koyun. Kapının yanındaki üçlü çekmecenin en alt çekmecesinde çöp poşetleri mevcut.
    \item \textbf{Çantalarınızı koltuk ve sandalyelere bırakmayın.} Tahtanın altındaki askılıklara asabilirsiniz. 
    \item Montları dolabın yanındaki askılığa asın. 
    \item Eşyalarınızı masada diğer insanların alanını işgal etmeyecek şekilde konumlandırın.
    \item Eğer odadan uzun bir süreliğine ayrılacaksanız eşylarınızı toplayın. Masayı ve koltuğu gereksiz yere meşgul etmeyin.
    \item Odada herhangi bir yerden aldığınız eşyayı geri \textbf{aynı yere} bırakın. Eğer aldığınız yeri hatırlayamıyorsanız, veya emin değilseniz, YK tarafından sağlanan oda eşya-konum listesine göre yerleştirebilirsiniz. Eğer hala emin değilseniz bir YK'ya sormaktan çekinmeyin.
    \item Özellikle kışın ve sonbaharda, ayakkabılarınıza çamur bulaşmışsa içeri taşımamak için oda girişindeki paspası kullanarak ayakkabılarınızı silmelisiniz.
\end{enumerate}

\section{Birlikte Yaşam}
\hspace{5mm} Her üyenin odadan rahatça yararlanabilmesi için son olarak bahsetmek istediğimiz birkaç kural daha var.
\begin{enumerate}
    \item 	Önceki iki bolümde bahsedilen kurallara ve \textbf{Code of Conduct}’a uyun.
    \item Odada sohbet ederken toplum içinde bulunduğunuzu unutmayın ve ses seviyenize dikkat edin.
    \item Odada ders çalışanlara saygı duyun ve rahatsız etmeyin. Odanın o anki durumunu sezebilin.
    \item Ses seviyesinden bağımsız bir şekilde diğer üyelerden izin almadan odada sesli yayın yapmayın.
    \item Eğer size YK tarafından anahtar emanet edildiyse, YK'nın bilgisi ve izni olmadan anahtarı başkasına veremezsiniz ve odada anahtarsız birisini/aktif olmayan herhangi birisini/birilerini bırakamazsınız. Çıkarken odayı \textbf{iki kere} kilitlemeniz gerekmektedir. 
\end{enumerate}

\end{document}